%% Generated by Sphinx.
\def\sphinxdocclass{report}
\documentclass[letterpaper,10pt,english,openany,oneside]{sphinxmanual}
\ifdefined\pdfpxdimen
   \let\sphinxpxdimen\pdfpxdimen\else\newdimen\sphinxpxdimen
\fi \sphinxpxdimen=.75bp\relax

\PassOptionsToPackage{warn}{textcomp}
\usepackage[utf8]{inputenc}
\ifdefined\DeclareUnicodeCharacter
% support both utf8 and utf8x syntaxes
  \ifdefined\DeclareUnicodeCharacterAsOptional
    \def\sphinxDUC#1{\DeclareUnicodeCharacter{"#1}}
  \else
    \let\sphinxDUC\DeclareUnicodeCharacter
  \fi
  \sphinxDUC{00A0}{\nobreakspace}
  \sphinxDUC{2500}{\sphinxunichar{2500}}
  \sphinxDUC{2502}{\sphinxunichar{2502}}
  \sphinxDUC{2514}{\sphinxunichar{2514}}
  \sphinxDUC{251C}{\sphinxunichar{251C}}
  \sphinxDUC{2572}{\textbackslash}
\fi
\usepackage{cmap}
\usepackage[T1]{fontenc}
\usepackage{amsmath,amssymb,amstext}
\usepackage{babel}



\usepackage{times}
\expandafter\ifx\csname T@LGR\endcsname\relax
\else
% LGR was declared as font encoding
  \substitutefont{LGR}{\rmdefault}{cmr}
  \substitutefont{LGR}{\sfdefault}{cmss}
  \substitutefont{LGR}{\ttdefault}{cmtt}
\fi
\expandafter\ifx\csname T@X2\endcsname\relax
  \expandafter\ifx\csname T@T2A\endcsname\relax
  \else
  % T2A was declared as font encoding
    \substitutefont{T2A}{\rmdefault}{cmr}
    \substitutefont{T2A}{\sfdefault}{cmss}
    \substitutefont{T2A}{\ttdefault}{cmtt}
  \fi
\else
% X2 was declared as font encoding
  \substitutefont{X2}{\rmdefault}{cmr}
  \substitutefont{X2}{\sfdefault}{cmss}
  \substitutefont{X2}{\ttdefault}{cmtt}
\fi


\usepackage[Bjarne]{fncychap}
\usepackage{sphinx}

\fvset{fontsize=\small}
\usepackage{geometry}


% Include hyperref last.
\usepackage{hyperref}
% Fix anchor placement for figures with captions.
\usepackage{hypcap}% it must be loaded after hyperref.
% Set up styles of URL: it should be placed after hyperref.
\urlstyle{same}

\usepackage{sphinxmessages}
\setcounter{tocdepth}{3}
\setcounter{secnumdepth}{3}


\title{VHH Plugin Package: Camera Movements Classification (vhh\_cmc)}
\date{Jun 30, 2021}
\release{1.2.2}
\author{Daniel Helm}
\newcommand{\sphinxlogo}{\vbox{}}
\renewcommand{\releasename}{Release}
\makeindex
\begin{document}

\pagestyle{empty}
\sphinxmaketitle
\pagestyle{plain}
\sphinxtableofcontents
\pagestyle{normal}
\phantomsection\label{\detokenize{index::doc}}


The following description gives an overview of the folder structure of this python repository:

\sphinxstyleemphasis{name of repository}: vhh\_cmc
\begin{itemize}
\item {} 
\sphinxstylestrong{ApiSphinxDocumentation/}: includes all files to generate the documentation as well as the created documentations (html, pdf)

\item {} 
\sphinxstylestrong{config/}: this folder includes the required configuration file

\item {} 
\sphinxstylestrong{cmc/}: this folder represents the shot\sphinxhyphen{}type\sphinxhyphen{}classification module and builds the main part of this repository

\item {} 
\sphinxstylestrong{Demo/}: this folder includes a demo script to demonstrate how the package have to be used in customized applications

\item {} \begin{description}
\item[{\sphinxstylestrong{Develop/}: includes scripts to generate the sphinx documentation. Furthermore, a script is included to run a}] \leavevmode
process to evaluate the implemented approach on a specified dataset.

\end{description}

\item {} 
\sphinxstylestrong{README.md}: this file gives a brief description of this repository (e.g. link to this documentation)

\item {} 
\sphinxstylestrong{requirements.txt}: this file holds all python lib dependencies and is needed to install the package in your own virtual environment

\item {} 
\sphinxstylestrong{setup.py}: this script is needed to install the cmc package in your own virtual environment

\end{itemize}


\chapter{Setup  instructions}
\label{\detokenize{index:setup-instructions}}
This package includes a setup.py script and a requirements.txt file which are needed to install this package for custom
applications. The following instructions have to be done to use this library in your own application:

\sphinxstylestrong{Requirements:}
\begin{itemize}
\item {} 
Ubuntu 18.04 LTS

\item {} 
python version 3.6.x

\end{itemize}

\sphinxstylestrong{Create a virtual environment:}
\begin{itemize}
\item {} 
create a folder to a specified path (e.g. /xxx/vhh\_cmc/)

\item {} 
python3 \sphinxhyphen{}m venv /xxx/vhh\_cmc/

\end{itemize}

\sphinxstylestrong{Activate the environment:}
\begin{itemize}
\item {} 
source /xxx/vhh\_cmc/bin/activate

\end{itemize}

\sphinxstylestrong{Checkout vhh\_cmc repository to a specified folder:}
\begin{itemize}
\item {} 
git clone \sphinxurl{https://github.com/dahe-cvl/vhh\_cmc}

\end{itemize}

\sphinxstylestrong{Install the cmc package and all dependencies:}
\begin{itemize}
\item {} 
change to the root directory of the repository (includes setup.py)

\item {} 
python setup.py install

\end{itemize}

\sphinxstylestrong{Setup environment variables:}
\begin{itemize}
\item {} 
source /data/dhelm/python\_virtenv/vhh\_sbd\_env/bin/activate

\item {} 
export CUDA\_VISIBLE\_DEVICES=1

\item {} 
export PYTHONPATH=\$PYTHONPATH:/XXX/vhh\_cmc/:/XXX/vhh\_cmc/Develop/:/XXX/vhh\_cmc/Demo/

\end{itemize}

\begin{sphinxadmonition}{note}{Note:}
You can check the success of the installation by using the commend \sphinxstyleemphasis{pip list}. This command should give you a list
with all installed python packages and it should include \sphinxstyleemphasis{vhh\_cmc}.
\end{sphinxadmonition}

\sphinxstylestrong{Run demo script}
\begin{itemize}
\item {} 
change to root directory of the repository

\item {} 
python Demo/vhh\_cmc\_run\_on\_single\_video.py

\end{itemize}


\chapter{Parameter Description}
\label{\detokenize{index:parameter-description}}
DEBUG\_FLAG
This parameter is used to activate or deactivate the debug mode.



SBD\_RESULTS\_PATH
This parameter is used to specify a SBD results file for debugging mode.



PATH\_DEBUG\_RESULTS
This parameter is used to specify the results path in debug mode



SAVE\_DEBUG\_PKG
This parameter is used to save a debug package (e.g. including some visualizations, … \sphinxhyphen{} not available yet).



CONVERT2GRAY\_FLAG
This flag is used to convert a input frame into a grayscale frame (0… deactivate, 1 … activate).



CENTER\_CROP\_FLAG
This flag is used to center crop a input frame (0… deactivate, 1 … activate).



DOWNSCALE\_FLAG
This flag is used to scale a input frame into the specified dimension (0… deactivate, 1 … activate).



RESIZE\_DIM
This flag is used to to specify the resize dimension. (only usable if DOWNSCALE\_FLAG is active).



MVI\_MV\_RATIO
This parameter is used to specify the ratio between available motion\sphinxhyphen{}vectors\sphinxhyphen{}of\sphinxhyphen{}interest to the all motion\sphinxhyphen{}vectors.



THRESHOLD\_SIGNIFICANCE
This parameter is used to specify the threshold (t1) for the significance check.



THRESHOLD\_CONSISTENCY
This parameter is used to specify the threshold (t2) for the consistency check.



MVI\_WINDOW\_SIZE
This parameter is used to specify the temporal window\_size (k) for the significance/consistency check.



REGION\_WINDOW\_SIZE
This parameter is used to specify the temporal window\_size (n) for the final movements classification over one shot.



ACTIVE\_THRESHOLD
This parameter is used to specify the percentage threshold to identify movement activities.



CLASS\_NAMES
This parameter is used to specify the class names.



SAVE\_RAW\_RESULTS
This parameter is used to save raw results (e.g. debug visualizations).



PATH\_RAW\_RESULTS
This parameter is used to specify the path for saving the raw results.



PREFIX\_RAW\_RESULTS
This parameter is used to specify the prefix for the results file.



POSTFIX\_RAW\_RESULTS
This parameter is used to specify the postfix for the results file.



SAVE\_FINAL\_RESULTS
This parameter is used to save final results (e.g. csv list).



PATH\_FINAL\_RESULTS
This parameter is used to specify the path for saving the final results.



PREFIX\_FINAL\_RESULTS
This parameter is used to specify the prefix for the results file.



POSTFIX\_FINAL\_RESULTS
This parameter is used to specify the postfix for the results file.



PATH\_VIDEOS
This parameter is used to specify the path to the videos.



SAVE\_EVAL\_RESULTS
This parameter is used to save evaluation results (e.g. visualizations, … ).



PATH\_RAW\_RESULTS
This parameter is used the raw results path.



PATH\_EVAL\_RESULTS
This parameter is used to specify the path to store the evaluation results path.



PATH\_GT\_ANNOTATIONS
This parameter is used to groundtruth annotations used for evaluation.



PATH\_EVAL\_DATASET
This parameter is used to specify the path to the dataset used for the evaluation.




\chapter{API Description}
\label{\detokenize{index:api-description}}
This section gives an overview of all classes and modules in \sphinxstyleemphasis{cmc} as well as an code description.


\section{Configuration class}
\label{\detokenize{Configuration:configuration-class}}\label{\detokenize{Configuration::doc}}

\section{CMC class}
\label{\detokenize{CMC:cmc-class}}\label{\detokenize{CMC::doc}}

\section{OpticalFlow class}
\label{\detokenize{OpticalFlow:opticalflow-class}}\label{\detokenize{OpticalFlow::doc}}

\section{PreProcessing class}
\label{\detokenize{PreProcessing:preprocessing-class}}\label{\detokenize{PreProcessing::doc}}

\section{Evaluation class}
\label{\detokenize{Evaluation:evaluation-class}}\label{\detokenize{Evaluation::doc}}

\chapter{Indices and tables}
\label{\detokenize{index:indices-and-tables}}\begin{itemize}
\item {} 
\DUrole{xref,std,std-ref}{genindex}

\item {} 
\DUrole{xref,std,std-ref}{modindex}

\item {} 
\DUrole{xref,std,std-ref}{search}

\end{itemize}


\section{References}
\label{\detokenize{index:references}}


\renewcommand{\indexname}{Index}
\printindex
\end{document}